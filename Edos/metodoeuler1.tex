\documentclass[11pt,a4paper]{article}

% ======== PAQUETES BÁSICOS ========
\usepackage[utf8]{inputenc}    % Acentos
\usepackage[T1]{fontenc}
\usepackage[spanish]{babel}    % Idioma español
\usepackage{amsmath, amssymb, amsfonts}  % Matemática
\usepackage{graphicx}          % Imágenes
\usepackage{physics}           % Derivadas, gradientes, etc.
\usepackage{siunitx}           % Unidades físicas (m, s, kg, etc.)
\usepackage{xcolor}            % Colores
\usepackage{geometry}          % Márgenes
\usepackage{fancyhdr}          % Encabezado/pie
\usepackage{hyperref}          % Enlaces
\usepackage{listings}          % Código fuente
\usepackage{caption}           % Subtítulos en código
\usepackage{tcolorbox}         % Cajas para resúmenes o ejemplos

% ======== CONFIGURACIÓN DE PÁGINA ========
\geometry{margin=2cm}
\pagestyle{fancy}
\fancyhf{}
\rhead{Apuntes de Física y Python}
\lhead{Isra — UdeC}
\cfoot{\thepage}

% ======== CONFIGURACIÓN DE CÓDIGO PYTHON ========
\definecolor{lightgray}{gray}{0.95}
\definecolor{darkgreen}{rgb}{0,0.5,0}
\lstset{
    language=Python,
    backgroundcolor=\color{lightgray},
    commentstyle=\color{darkgreen},
    keywordstyle=\color{blue}\bfseries,
    stringstyle=\color{red},
    basicstyle=\ttfamily\footnotesize,
    frame=single,
    numbers=left,
    numberstyle=\tiny\color{gray},
    breaklines=true,
    showstringspaces=false,
    tabsize=4
}

% ======== ESTILO DE CAJAS ========
\tcbset{
    colback=blue!5!white,
    colframe=blue!70!black,
    fonttitle=\bfseries,
    coltitle=black,
    boxrule=0.8pt,
    arc=3pt
}

% ======== INICIO DEL DOCUMENTO ========
\begin{document}

\begin{center}
    {\LARGE \textbf{Metodo de Euler/Runge-Kutta}}\\[4pt]
    {\large Universidad de Concepción — Israel Bravo}\\[10pt]
    \hrulefill
\end{center}

\section*{Metodo de Euler}
Para empezar con el metodo de Euler tenemos que establecer la forma general de las ecuaciones que resolveremos con este metodo:

\begin{equation}
	\frac{du}{dt} = f(u,t)
\end{equation}

al encontrar cualquier funcion de este tipo, el metodo de euler nos dice que:

\begin{equation}
	u_{n+1}=f_n+f(u_n,t_n) \Delta t
\end{equation}

Podemos hacer lo siguiente en python para programar un metodo de resolver edos de la siguiente manera:

\begin{lstlisting}[caption={Programar el metodo de Euler}]
#Primero que nada, definimos el metodo de euler como:

def euler_system(f, u, t):
    u = np.copy(u)
    dt = t[1] - t[0]
    for n in range(len(t)-1):
        u[n+1, :] = u[n, :] + f(u[n, :], t[n]) * dt
    return u

\end{lstlisting}
Con esto definido ya podemos empezar a usar el metodo de euler de la siguiente manera:

\begin{lstlisting}[caption={Definamos la funcion y todos los parametros}]
#Importemos los modulos numpy y ademas programemos la edo que resolveremos:
import numpy as np
import matplotlib.pyplot as plt

def lotka_volterra(u,t)
	x,y = u
	alpha = 1.0
	beta = 0.1
	delta = 0.075
	gamma = 1.5
	dxdt = alpha*x-beta*x*y
	dydt = delta*x*y - gamma*y
	return np.array([dxdt,dydt])



\end{lstlisting}

\section*{Notas rápidas}
\begin{itemize}
    \item $\dv{y}{t} = v_0 \sin(\theta) - g t$ describe la velocidad vertical.
    \item El tiempo total de vuelo es $T = \frac{2v_0\sin(\theta)}{g}$.
\end{itemize}

\end{document}

